%%%%%%%%%%%%%%%%%%%%%%%%%%%%%%%%%%%%%%%%%
% Jacobs Landscape Poster
% LaTeX Template
% Version 1.1 (14/06/14)
%
% Created by:
% Computational Physics and Biophysics Group, Jacobs University
% https://teamwork.jacobs-university.de:8443/confluence/display/CoPandBiG/LaTeX+Poster
% 
% Further modified by:
% Nathaniel Johnston (nathaniel@njohnston.ca)
%
% This template has been downloaded from:
% http://www.LaTeXTemplates.com
%
% License:
% CC BY-NC-SA 3.0 (http://creativecommons.org/licenses/by-nc-sa/3.0/)
%
%%%%%%%%%%%%%%%%%%%%%%%%%%%%%%%%%%%%%%%%%

%----------------------------------------------------------------------------------------
%	PACKAGES AND OTHER DOCUMENT CONFIGURATIONS
%----------------------------------------------------------------------------------------

\documentclass[final,table]{beamer}

\usepackage[scale=1.24]{beamerposter} % Use the beamerposter package for laying out the poster

\usetheme{confposter} % Use the confposter theme supplied with this template

\setbeamercolor{block title}{fg=ngreen,bg=white} % Colors of the block titles
\setbeamercolor{block body}{fg=black,bg=white} % Colors of the body of blocks
\setbeamercolor{block alerted title}{fg=white,bg=dblue!70} % Colors of the highlighted block titles
\setbeamercolor{block alerted body}{fg=black,bg=dblue!10} % Colors of the body of highlighted blocks
% Many more colors are available for use in beamerthemeconfposter.sty

%-----------------------------------------------------------
% Define the column widths and overall poster size
% To set effective sepwid, onecolwid and twocolwid values, first choose how many columns you want and how much separation you want between columns
% In this template, the separation width chosen is 0.024 of the paper width and a 4-column layout
% onecolwid should therefore be (1-(# of columns+1)*sepwid)/# of columns e.g. (1-(4+1)*0.024)/4 = 0.22
% Set twocolwid to be (2*onecolwid)+sepwid = 0.464
% Set threecolwid to be (3*onecolwid)+2*sepwid = 0.708

\newlength{\sepwid}
\newlength{\onecolwid}
\newlength{\twocolwid}
\newlength{\threecolwid}
\setlength{\paperwidth}{48in} % A0 width: 46.8in
\setlength{\paperheight}{36in} % A0 height: 33.1in
\setlength{\sepwid}{0.024\paperwidth} % Separation width (white space) between columns
\setlength{\onecolwid}{0.22\paperwidth} % Width of one column
\setlength{\twocolwid}{0.464\paperwidth} % Width of two columns
\setlength{\threecolwid}{0.708\paperwidth} % Width of three columns
\setlength{\topmargin}{-0.5in} % Reduce the top margin size
%-----------------------------------------------------------
\usepackage{etex}
\usepackage{graphicx}  % Required for including images
\usepackage{booktabs} % Top and bottom rules for tables
\usepackage{tikz}
\usepackage{blkarray}
\usepackage{array}
\usepackage{multirow}
\usepackage{textcomp}
\usepackage{multirow}
\usepackage{comment}
\usepackage{amsmath}
\usepackage{amsthm} 
\usepackage{amssymb}
\usepackage{ifthen}
\usepackage{pgfplots}
\usepackage{caption}
\usepackage{subcaption}
\usepackage{relsize}
\usepackage{times}
\usepackage{url}
\usepackage{latexsym}
\usepackage{graphicx}
\usepackage{ifthen}
\usepackage{comment}
%\usepackage{floatrow}
\usepackage{algorithm}
\usepackage{algpseudocode}
\usepackage{multirow}
\usepackage{xcolor}
\captionsetup{compatibility=false}


\usetikzlibrary{decorations.shapes,shapes.multipart,matrix,positioning,arrows,intersections,tikzmark,patterns,fit,calc,arrows.meta}

% Paper-specific macros

\newcommand\citep\cite
\newcommand\citet\newcite

\newcommand\pl[1]{\textcolor{red}{[PL: #1]}}
\newcommand\eng{\textcolor{red}{\pl{fix English}}}

\newcommand\pointM{M_p}
\newcommand\ourM{\hat M}
\newcommand{\Cext}{\sC^+}

\newcommand\fk[1]{\textcolor{blue}{[FK: #1]}}
\newcommand\nl[1]{\emph{#1}}
\newcommand\nlq[1]{``\emph{#1}''}
\newcommand\wl[1]{\texttt{#1}}
\newcommand\wlq[1]{``\texttt{#1}''}

\newcommand{\sClp}{\sC_\text{\rm LP}}
\newcommand{\sCls}{\sC_\text{\rm LS}}
\newcommand{\sFlp}{\sF_\text{\rm LP}}
\newcommand{\sFls}{\sF_\text{\rm LS}}


\newcommand{\SSe}{\sigma}
\newcommand{\TSe}{\tau}
\newcommand{\ns}{n_\text{s}}
\newcommand{\nt}{n_\text{t}}
\newcommand{\SF}{S}
\newcommand{\TF}{T}
\newcommand{\nothing}{$\emptyset$}
\newcommand{\sep}{-}
\newcommand{\pZ}{\Z_{\scriptscriptstyle\geq0}}
\newcommand{\pR}{\R_{\scriptscriptstyle\geq0}}
\newcommand{\nmistakes}{n_\text{mistakes}}

\newcommand{\matindex}[1]{\mbox{\scriptsize#1}}% Matrix index
\newcommand{\matindexw}[1]{\mbox{\begin{sideways}\scriptsize#1\end{sideways}}}% Matrix index



%\renewcommand{\thefootnote}{\fnsymbol{footnote}}

\input{std-macro}
%----------------------------------------------------------------------------------------
%	TITLE SECTION 
%----------------------------------------------------------------------------------------

\title{\fontsize{75}{0}\selectfont Unanimous Prediction For 100\% Precision With Application To Learning Semantic Mappings} % Poster title

\author{Fereshte Khani, Martin Rinard, Percy Liang} % Author(s)

\institute{Stanford University} % Institution(s)

%----------------------------------------------------------------------------------------

\begin{document}
\addtobeamertemplate{headline}{} 
{
\begin{tikzpicture}[remember picture,overlay] 
\node [shift={(6 cm,-9cm)}] at (current page.north west) {\includegraphics[height=5cm]{nlp-logo-10x10.jpg}}; 
\end{tikzpicture} 
}

\addtobeamertemplate{block end}{}{\vspace*{2ex}} % White space under blocks
\addtobeamertemplate{block alerted end}{}{\vspace*{2ex}} % White space under highlighted (alert) blocks

\setlength{\belowcaptionskip}{2ex} % White space under figures
\setlength\belowdisplayshortskip{2ex} % White space under equations

\begin{frame}[t] % The whole poster is enclosed in one beamer frame

\begin{columns}[t] % The whole poster consists of three major columns, the second of which is split into two columns twice - the [t] option aligns each column's content to the top

\begin{column}{\sepwid}\end{column} % Empty spacer column

\begin{column}{\onecolwid} % The first column

%----------------------------------------------------------------------------------------
%	INTRODUCTION
%----------------------------------------------------------------------------------------

\begin{block}{Introduction}
If a user asks a system \nlq{How many painkillers should I take?},
%In user-facing or safety-critical applications,
it is much better for the system to say ``don't know''
rather than making a costly incorrect prediction.
\end{block}
%----------------------------------------------------------------------------------------
%	Analogy
%----------------------------------------------------------------------------------------

\begin{block}{Analogy}


\begin{figure}\centering

\tikzstyle{bluePoint} = [circle,fill=blue,minimum size=15pt, inner sep=0pt]
\tikzstyle{redPoint} = [circle, fill=red, minimum size =15pt, inner sep=0pt]
\tikzstyle{line} = [draw,thick,-]



\begin{tikzpicture}[auto,swap]


	\coordinate (A) at (0,0);
	\coordinate (B) at (1,3);
	\coordinate (C) at (3,1);
	\newcommand{\dis}{10}
	\coordinate (D) at (\dis + 0,0);
	\coordinate (E) at (\dis + 1,3);
	\coordinate (F) at (\dis + 3,1);
	
	
	\node[bluePoint] (Ap) at (A) {};
	\node[bluePoint] (Bp) at (B) {};
	\node[bluePoint] (Cp) at (C) {};
		
	\node[redPoint] (Dp) at (D) {};
	\node[redPoint] (Ep) at (E) {};
	\node[redPoint] (Fp) at (F) {};
	
	\draw  (-3,-3) rectangle (16,6);
	
	\draw (8,-3) -- (8,6);
	\draw (7,-3) -- (7,6);	
	\draw (6,-3) -- (6,6);
	\draw (9,-3) -- (9,6);
	\draw (5,-3) -- (5,6);
	\draw (4,-3) -- (4,6);

	\draw (2,-3) -- (10,6);
	\draw (1,-3) -- (9,6);
	\draw (3,-3) -- (11,6);
	\draw (2,-3) -- (8,6);
	\draw (2,-3) -- (7,6);
	\draw (4,-3) -- (11,6);
	\draw (4,-3) -- (11,6);
	\draw (0,-3) -- (13,6);

	\draw (12,-3) -- (0,6);
	\draw (11,-3) -- (-1,6);
	\draw (13,-3) -- (1,6);
	\draw (12,-3) -- (-2,6);
	\draw (12,-3) -- (-3,6);
	\draw (13.5,-3) -- (1,6);
	\draw (14,-3) -- (0,6);
	\draw (10,-3) -- (3,6);

	\fill[blue,opacity=0.2] (-3,-1) -- (A) -- (C) -- (B) -- (-3,4.5)--cycle;
	
	\fill[red,opacity=0.2] (16,-2) -- (D) -- (E)  -- (16,5)--cycle;
	
	\node (always blue) at (0,2) {\small always blue};
	\node (always red) at (\dis+3, 2) {\small always red};
%	\node (cannot guarantee) at (\dis/2+1, 4) {\small cannot guarantee};

\end{tikzpicture}




\end{figure}
\end{block}

%----------------------------------------------------------------------------------------
%	Goal
%----------------------------------------------------------------------------------------

\begin{block}{Goal}
We present a system which {\bf learns}
a {\color{BlueViolet} semantic mapping} which {\bf guarantees 100\% precision} under its model assumptions.

\centering
 {\color{BlueViolet} \nl{area of Ohio} $\rightarrow$ \wl{\{area, OH\}}}
\end{block}

\begin{block}{Unanimity principle}
\begin{figure}
	\newcommand{\scale}{0.6}
	\tikzset{
    >=stealth',
    inputBox/.style={
           rectangle,
			inner sep=0,
           rounded corners=0.1em,
           draw=black,
           text centered,
			scale=\scale
},
   mappingBox/.style={
           rectangle,
			inner sep=0,
           rounded corners=0.3em,
           draw=black,
			fill=gray!30,
           text centered,
			scale=\scale
},
    pil/.style={
line width=6,
           ->,
           thick,
		}
}

		
			\begin{tikzpicture}%[scale=\scale]
			
			
			
			\def\x{4.4}
			\def\y{3}
			\def\yy{2.5}
			
			
			\def\spx{0.2}
			\def\spy{0.2}
			
			
			\node (trainings)[inputBox] at (1.5*\x,\y+\yy+4*\spy){
				\begin{tabular}{l|l}
				\bf input & \bf output \\ 	\hline 
				\nl{area of Iowa}& \wl{area(IA)}\\
				\nl{cities in Ohio}&  \wl{city(OH)} \\
				\nl{cities in Iowa} & \wl{city(IA)}\\

				\end{tabular}
			};
			
			\node (A)[mappingBox] at (0,\yy){
				\begin{tabular}{l}
				\bf mapping 1 \\ 	
				\end{tabular}
			};


			
			\node (B) [mappingBox]at (\x,\yy){
				\begin{tabular}{l}
				\bf mapping 2 \\				
				\end{tabular}
			};
			
			\node (C) [mappingBox]at (2*\x+4*\spx,\yy){
				\begin{tabular}{l}
				 \bf mapping k \\ 
				\end{tabular}
			};

				\node (AA)[inputBox] at (0,0){
				\begin{tabular}{l}
				\bf output 1 \\ \hline
				\small\wl{area(OH)}\\
				\small\wl{area(OH)}\\
				%\small\wl{city} \\
				\end{tabular}
			};
			
			\node (BB) [inputBox]at (\x,0){
				\begin{tabular}{l}
				\bf output 2 \\ \hline
				\small\wl{area(OH)}\\
				\small\wl{area(OH)}\\
				%\small\wl{city} \\
				\end{tabular}
			};
			
			\node (CC) [inputBox]at (2*\x+4*\spx,0){
				\begin{tabular}{l}
				\bf output k \\ \hline
				\small\wl{area(OH)}\\
				\small\wl{OH}\\
				%\small\wl{city} \\
				\end{tabular}
			};
			
			
			\node [left] at (trainings.west) {\tiny training examples};
			
			
			\node (testingsInput)[inputBox] at (-1*\x-3*\spx-\spx,0){
				\begin{tabular}{l}
				\bf input \\ \hline
				\small\nl{area of Ohio}\\ 
				\small\nl{Ohio area}\\ 
				%\small\nl{cities in}\\ \hline
				\end{tabular}
			};
			
			\node (testingsOutput)[inputBox] at (4*\x+4*\spx,0){
				\begin{tabular}{l}
				\bf output \\ \hline
				\small\wl{area(Ohio)}\\ 
				\small don't know \\ 
				%\small \wl{city} \\ \hline
				\end{tabular}
			};
			
			\node [above] at (testingsInput.north) {\tiny testing examples};
			
			%\node [above] at (testingsOutput.north) {\tiny test examples};
			
			\def\checkmark{\tikz\fill[fill=green,scale=1.5](0,.35) -- (.25,0) -- (1,.7) -- (.25,.15) -- cycle;} 
			
			\newcommand{\Cross}{$\mathbin{\tikz [line width=.5ex, red,scale=0.75] \draw (0,0) -- (1,1) (0,1) -- (1,0);}$}%
			
			\node (testingsResults)[scale=\scale] at (3*\x + 2*\spx,0){
				\begin{tabular}{c}
				\small \bf unanimity \\ 
				\checkmark\\
				\Cross\\
			%	\checkmark\\
				\end{tabular}
			};
			
			\node (3dots)[] at (1.5*\x+2*\spx,\yy){...};
			\node (3dots)[] at (1.5*\x+2*\spx,0){...};
			
    \node[fit={(-\x/2-\spx,-\y/2-2*\spy) (3.6*\x,\y/2+\yy-\spy)}, inner sep=0pt, draw=black,rounded corners=.3em] (rect) {};

		%	\node (rec)[scale=\scale] { (-\x/2,-\y/2) rectangle (3.6*\x,\y/2+\yy)};
			
			\draw[pil, shorten >=3pt] (rect.north) -- (A.north);
			\draw[pil, shorten >=2pt] (rect.north) -- (B.north);
			\draw[pil, shorten >=3pt] (rect.north) -- (C.north);
			
			\draw[pil] (trainings.south) -- (rect.north);
			
			\draw[pil] (3.6*\x,0) -- (testingsOutput.west);
			\draw[pil] (testingsInput.east) -- (-\x/2-\spx, 0);
			
			
			\draw[double,-implies] (A.south) -- (AA.north);
			\draw[double,-implies] (B.south) -- (BB.north);
			\draw[double,-implies] (C.south) -- (CC.north);
			
		

\end{tikzpicture}

\end{figure}



\begin{block}{Framework}
\begin{itemize}
\item[] {\bf Training set:}
\begin{align*}
\{(x_1,y_1),(x_2,y_2),\dots, (x_n,y_n)\}
\end{align*} 
\begin{itemize}
\item[] Source atoms
\newcommand{\scale}{0.75}
\tikzset{
    %Define standard arrow tip
    >=stealth',
    %Define style for boxes
    punkt/.style={
           rectangle,
           rounded corners,
           draw=black, very thick,
           text width=6.5em,
           minimum height=2em,
           text centered},
    % Define arrow style
    pil/.style={
line width=6,
           ->,
           thick,
           scale=8,
		}
}

\begin{center}
\begin{tikzpicture}

\node (trainings)[scale=\scale] at (0,0){
	\begin{tabular}{|c|}
	\hline 
	{\bf utterences} \\ \hline
	\nl{area of Iowa}\\ \hline
	\nl{cities in Ohio}\\ \hline
	\nl{cities in Iowa}\\ \hline
	\end{tabular}
};

\node (sourceAtoms)[scale=\scale] at (10,0){
	\begin{tabular}{|c|}
	\hline
	 \bf source atoms \\ \hline
	\{\nl{area}, \nl{of}, \nl{Iowa}\}\\ \hline
	\{\nl{cities}, \nl{in}, \nl{Ohio}\}\\ \hline
	\{\nl{cities}, \nl{in}, \nl{Iowa}\} \\ \hline
	\end{tabular}
};

\draw[pil] (trainings.east) -- (sourceAtoms.west);
\end{tikzpicture}
\end{center}



\item[] Target atoms

\begin{center}

	\begin{tikzpicture}

	\node (trainings)[scale=\scale] at (0,0){
		\begin{tabular}{|c|}
		\hline 
		\bf logical forms \\ \hline
		\wl{area(IA)}\\ \hline
		\wl{city(OH)} \\ \hline
		\wl{city(IA)}\\ \hline
		\end{tabular}
	};
	
	\node (targetAtoms)[scale=\scale] at (10,0){
		\begin{tabular}{|c|}
		\hline
		\bf target atoms \\ \hline
		\{\wl{area}, \wl{IA}\}\\ \hline
		\{\wl{city}, \wl{OH}\} \\ \hline
		\{\wl{city}, \wl{IA}\}\\ \hline
		\end{tabular}
	};
	\draw[pil] (trainings.east) -- (targetAtoms.west);
	\end{tikzpicture}
\end{center}



\end{itemize}
\end{itemize}
\end{block}
\end{block}

\end{column} % End of the first column

\begin{column}{\sepwid}\end{column} % Empty spacer column

\begin{column}{\onecolwid} % Begin a column which is two columns wide (column 2)


%----------------------------------------------------------------------------------------
%	MATERIALS
%----------------------------------------------------------------------------------------

\begin{block}{Framework}
\begin{itemize}
\item[] {\bf Hypothesis space ($\sM$):}
\vspace{1cm}

\begin{tikzpicture}
\newcommand{\sscale}{0.6}
	\node (A)[scale=\sscale] at (0,0){
		\begin{tabular}{|l|}
		\hline \bf mapping 1 \\ \hline
		 \nl{cities} $\rightarrow$  \{\wl{city}\} \\
		 \nl{in} $\rightarrow$  \{\} \\
		 \nl{of} $\rightarrow$  \{\} \\
		 \nl{area} $\rightarrow$  \{\wl{area}\} \\
		 \nl{Iowa} $\rightarrow$  \{\wl{IA}\} \\
		\nl{Ohio} $\rightarrow$  \{\wl{OH}\} \\
		\hline
		\end{tabular}
	};
	
	\node (B) [scale=\sscale]at (5.5,0){
		\begin{tabular}{|l|}
		\hline \bf mapping 2 \\ \hline
		\nl{cities} $\rightarrow$  \{\} \\
		\nl{in} $\rightarrow$  \{\} \\
		\nl{of} $\rightarrow$  \{\} \\
		\nl{area} $\rightarrow$  \{\} \\
		\nl{Iowa} $\rightarrow$  \{\} \\
		\nl{Ohio} $\rightarrow$  \{\} \\
		\hline
		\end{tabular}
	};
	
	\node (C) [scale=\sscale]at (16.5,0){
		\begin{tabular}{|l|}
		\hline \bf mapping k \\ \hline
		\nl{cities} $\rightarrow$  \{\wl{city,area,IA,OH}\} \\
		\nl{in} $\rightarrow$  \{\} \\
		\nl{of} $\rightarrow$  \{\} \\
		\nl{area} $\rightarrow$  \{\} \\
		\nl{Iowa} $\rightarrow$  \{\} \\
		\nl{Ohio} $\rightarrow$  \{\wl{area,area,city,city}\} \\
		\hline
		\end{tabular}
	};


\node (3dots)[] at (9,0){...};

	

      
 
\end{tikzpicture}



%Assumption: $M^* \in \sM$.
\vspace{1cm}
\item[] {\bf Consistent mappings ($\sC$):} 
\vspace{1cm}
\begin{align*} 
\sC \eqdef \{ M \in \sM \mid \forall i, M(x_i) = y_i\}
\end{align*}

\begin{tikzpicture}
\tikzset{
    myarrow/.style={->, >=latex', shorten >=2pt},
	mybox/.style={
	scale=\scale,
%	rounded corners=.1em, 
	draw=black,
	inner sep=1,
%	text width=3cm,
},
}
\def\x{6.6}
\def\y{2.5}
\def\btw{5}
\def\scale{0.62}

	\node (A)[mybox] at (0,\y){
		\begin{tabular}{lll}
		\multicolumn{3}{c}{\bf mapping 1} \\ \hline
		 \nl{cities}&$\rightarrow$&\{\wl{city}\} \\
		 \nl{in}&$\rightarrow$&\{\} \\
		 \nl{of}&$\rightarrow$  &\{\} \\
		 \nl{area}&$\rightarrow$ & \{\wl{area}\} \\
		 \nl{Iowa}&$\rightarrow$ & \{\wl{IA}\} \\
		\nl{Ohio}&$\rightarrow$ & \{\wl{OH}\} \\
		\end{tabular}
	};

	
	\node (B) [mybox]at (\x,\y){
			\begin{tabular}{lll}
		\multicolumn{3}{c}{\bf mapping 2} \\ \hline
		 \nl{cities}&$\rightarrow$&\{\wl{}\} \\
	 \nl{in}&$\rightarrow$&\{\wl{city}\} \\
		 \nl{of}&$\rightarrow$  &\{\} \\
		 \nl{area}&$\rightarrow$ & \{\wl{area}\} \\
		 \nl{Iowa}&$\rightarrow$ & \{\wl{IA}\} \\
		\nl{Ohio}&$\rightarrow$ & \{\wl{OH}\} \\
		\end{tabular}
	};

	
	\node (C) [mybox]at (2*\x ,\y){
			\begin{tabular}{lll}
		\multicolumn{3}{c}{\bf mapping 3} \\ \hline
		 \nl{cities}&$\rightarrow$&\{\wl{city}\} \\
	 \nl{in}&$\rightarrow$&\{\} \\
		 \nl{of}&$\rightarrow$  &\{\wl{area}\} \\
		 \nl{area}&$\rightarrow$ & \{\wl{}\} \\
		 \nl{Iowa}&$\rightarrow$ & \{\wl{IA}\} \\
		\nl{Ohio}&$\rightarrow$ & \{\wl{OH}\} \\
		\end{tabular}
	};


	\node (D) [mybox]at (3*\x ,\y){
		\begin{tabular}{lll}
		\multicolumn{3}{c}{\bf mapping 4} \\ \hline
		 \nl{cities}&$\rightarrow$&\{\wl{}\} \\
	 \nl{in}&$\rightarrow$&\{\wl{city}\} \\
		 \nl{of}&$\rightarrow$  &\{\wl{area}\} \\
		 \nl{area}&$\rightarrow$ & \{\wl{}\} \\
		 \nl{Iowa}&$\rightarrow$ & \{\wl{IA}\} \\
		\nl{Ohio}&$\rightarrow$ & \{\wl{OH}\} \\
		\end{tabular}
	};
      

\end{tikzpicture}


\vspace{1cm}
\item[] {\bf Safe set ($\sF$):} 
\vspace{1cm}
\begin{align*}
\sF \eqdef \{ x : |\{ M(x) : M \in \sC \}| = 1 \}
\end{align*}
\centering
\begin{tikzpicture}
\newcommand{\sscale}{0.6}

	
	\node (B) [scale=\sscale,anchor=north west]at (0,0){
		\begin{tabular}{|c|}
		\hline \bf safe set  \\ \hline
		\nl{Ohio}\\
		\nl{cities in}\\
		\nl{area of Ohio}\\
		\nl{area of cities in}\\
		\\ \\
		\hline
		\end{tabular}
	};
%I really couldn't fit 3dots or vdots or multicolumns here :( I think there should be something with vdots!!!
\fill (3,-5.5) circle (0.1);  
\fill (3,-6) circle (0.1);
\fill (3,-6.5) circle (0.1);
	
	\node (C) [scale=\sscale,anchor=north west]at (10,0){
		\begin{tabular}{|c|}
		\hline \bf not in the safe set \\ \hline
		\nl{Texas}\\
		\nl{Ohio area}\\
		\nl{Iowa cities} \\
	\nl{rivers in Texas area}\\
		\\ \\
		\hline
		\end{tabular}
	};

\fill (13.5,-5.5) circle (0.1);
\fill (13.5,-6)   circle (0.1);
\fill (13.5,-6.5) circle (0.1);

\end{tikzpicture}

\end{itemize}
\end{block}
\begin{block}{Linear algebraic formulation}

\centering
\scalebox{0.8}{
			$\overbrace{\begin{blockarray}{lcccccc}
			&\matindex{\nl{area}} & \matindex{\nl{of}} & \matindex{\nl{Ohio}} & \matindex{\nl{cities}} & \matindex{\nl{in}} & \matindex{\nl{Iowa}}\\	
			\begin{block}{l[cccccc]}
			\matindex{\nl{area of Iowa}} & 1 & 1 & 0 & 0 & 0 & 1 \\
			\matindex{\nl{cities in Ohio}} & 0 & 0 & 1 & 1 & 1 & 0 \\
			\matindex{\nl{cities in Iowa}} & 0 & 0 & 0 & 1 & 1 & 1 \\
			\end{block}
			\end{blockarray}}^{\mathlarger{\mathlarger{\mathlarger{\bf S }}}}$}\hspace{2cm}
			\scalebox{0.8}{	$\overbrace{\begin{blockarray}{lcccc}
			&\matindex{\wl{area}} & \matindex{\wl{city}} & \matindex{\wl{OH}} & \matindex{\wl{IA}}\\	
			\begin{block}{l[cccc]}
			\matindex{\wl{area(IA)}} & 1 & 0 & 0 & 1\\
			\matindex{\wl{city(OH)}} & 0 & 1 & 1 & 0\\
			\matindex{\wl{city(IA)}} & 0 & 1 & 0 & 1\\
			\end{block}
			\end{blockarray}}^{\mathlarger{\mathlarger{\mathlarger{\bf T }}}}$}


\centering
\begin{tikzpicture}
	
	\node (AA)[scale=0.70] at (0,-7){
		{\Large M = }$\begin{blockarray}{ccccc}
		&\matindex{\wl{area}} & \matindex{\wl{city}} & \matindex{\wl{OH}} &  \matindex{\wl{IA}}\\	
		\begin{block}{c[cccc]}
		\matindex{\nl{area}}		& 1 & 0 & 0 & 0\\
		\matindex{\nl{of}} 		& 0 & 0 & 0 & 0\\
		\matindex{\nl{Ohio}}		& 0 & 0 & 1 & 0 \\
		\matindex{\nl{cities}}	& 0 & 1 & 0 & 0 \\
		\matindex{\nl{in}}		& 0 & 0 & 0 & 0\\
		\matindex{\nl{Iowa}}		& 0 & 0 & 0 & 1\\
		\end{block}
		\end{blockarray}
		$\quad{\Huge \dots}
	};
\begin{comment}	
	\node (BB) [scale=0.70]at (7,-7){
		$\begin{blockarray}{ccccc}
		&\matindex{\wl{area}} & \matindex{\wl{city}} & \matindex{\wl{OH}} &  \matindex{\wl{IA}}\\	
		\begin{block}{c[cccc]}
		\matindex{\nl{area}}	& 0 & 0 & 0 & 0\\
		\matindex{\nl{of}} 		& 0 & 0 & 0 & 0\\
		\matindex{\nl{Ohio}}	& 0 & 0 & 0 & 0 \\
		\matindex{\nl{cities}}	& 0 & 0 & 0 & 0 \\
		\matindex{\nl{in}}		& 0 & 0 & 0 & 0\\
		\matindex{\nl{Iowa}}	& 0 & 0 & 0 & 0\\
		\end{block}
		\end{blockarray}
		$
	};
\end{comment}	
	\node (CC) [scale=0.70]at (10,-7){
		$\begin{blockarray}{ccccc}
		&\matindex{\wl{area}} & \matindex{\wl{city}} & \matindex{\wl{OH}} &  \matindex{\wl{IA}}\\	
		\begin{block}{c[cccc]}
		\matindex{\nl{area}}	& 1 & 1 & 1 & 1\\
		\matindex{\nl{of}} 		& 0 & 0 & 0 & 0\\
		\matindex{\nl{Ohio}}	& 0 & 0 & 0 & 0 \\
		\matindex{\nl{cities}}	& 0 & 0 & 0 & 0 \\
		\matindex{\nl{in}}		& 0 & 0 & 0 & 0\\
		\matindex{\nl{Iowa}}	& 2 & 2 & 0 & 0\\
		\end{block}
		\end{blockarray}
		$
	};


	

      
 
\end{tikzpicture}




\vspace{1cm}
\centering
$xM=y$
		\scalebox{0.8}{$
			\begin{blockarray}{ccccccc}
			\matindex{\nl{area}} & \matindex{\nl{of}} & \matindex{\nl{Ohio}} & \matindex{\nl{cities}} & \matindex{\nl{in}} & \matindex{\nl{Iowa}}&\\	
			\begin{block}{[cccccc]c}
			1 & 1 & 0 & 0 & 0 & 1 & \times\\
			\end{block}
			\end{blockarray}
		\begin{blockarray}{ccccc}
				&\matindex{\wl{area}} & \matindex{\wl{city}} & \matindex{\wl{OH}} & \matindex{\wl{IA}}\\		
				\begin{block}{c[cccc]}
				\matindex{\nl{area}}	& 1 & 0 & 0 & 0\\
				\matindex{\nl{of}} 		& 0 & 0 & 0 & 0\\
				\matindex{\nl{Ohio}}	& 0 & 0 & 1 & 0 \\
				\matindex{\nl{cities}}	& 0 & 1 & 0 & 0 \\
				\matindex{\nl{in}}		& 0 & 0 & 0 & 0\\
				\matindex{\nl{Iowa}}	& 0 & 0 & 0 & 1\\
				\end{block}
				\end{blockarray} 			
			\begin{blockarray}{ccccc}
			&\matindex{\wl{area}} & \matindex{\wl{city}} & \matindex{\wl{OH}} & \matindex{\wl{IA}}\\	
			\begin{block}{c[cccc]}
			= & 1 & 0 & 0 & 1\\
			\end{block}
			\end{blockarray}
			$}


\end{block}

\end{column} % End of column 2.1
\begin{column}{\sepwid}\end{column} % Empty spacer column
\begin{column}{\onecolwid} % The second column within column 2 (column 2.2)

%----------------------------------------------------------------------------------------
%	MATHEMATICAL SECTION
%----------------------------------------------------------------------------------------

\centering
$SM=T$
	\begin{center}
		\scalebox{0.65}{$
			\overbrace{\begin{blockarray}{cccccc}
			\matindex{\nl{area}} & \matindex{\nl{of}} & \matindex{\nl{Ohio}} & \matindex{\nl{cities}} & \matindex{\nl{in}} & \matindex{\nl{Iowa}}\\	
			\begin{block}{[cccccc]}
			1 & 1 & 0 & 0 & 0 & 1 \\
			0 & 0 & 1 & 1 & 1 & 0 \\
			0 & 0 & 0 & 1 & 1 & 1 \\
			\end{block}
			\end{blockarray}}^{S} \times
		\overbrace{\begin{blockarray}{ccccc}
		&\matindex{\wl{area}} & \matindex{\wl{city}} & \matindex{\wl{OH}} &  \matindex{\wl{IA}}\\	
		\begin{block}{c[cccc]}
		\matindex{\nl{area}} 	 &\multicolumn{4}{c}{} \\
		\matindex{\nl{of}} 		 &\multicolumn{4}{c}{\multirow{4}{*}{\Huge ?}} \\
		\matindex{\nl{Ohio}}	 &\multicolumn{4}{c}{} \\
		\matindex{\nl{cities}}	 &\multicolumn{4}{c}{} \\
		\matindex{\nl{in}}		 &\multicolumn{4}{c}{} \\
		\matindex{\nl{Iowa}}	 &\multicolumn{4}{c}{} \\
		\end{block}
		\end{blockarray}}^{M} = 			
			\overbrace{\begin{blockarray}{cccc}
			\matindex{\wl{area}} & \matindex{\wl{city}} & \matindex{\wl{OH}} & \matindex{\wl{IA}}\\	
			\begin{block}{[cccc]}
			1 & 0 & 0 & 1\\
			0 & 1 & 1 & 0\\
			0 & 1 & 0 & 1\\
			\end{block}
			\end{blockarray}}^{T}
			$}
	\end{center}

	

\centering
\vspace{2cm}


\begin{tikzpicture}
\draw (0,0) -- (2,3);
\draw (0,2) -- (6,2);
\draw (3,3) -- (6,0);

\fill[green!30] (0,0) -- (1.3,2) -- (4,2) -- (6,0) -- cycle;
\draw[thick] (0,0) -- (2,3);
\draw (0,2) -- (6,2);
\draw (3,3) -- (6,0);

\node (c) at (3,1) {$\sC$};

\end{tikzpicture}\hspace{3cm}%
\begin{tikzpicture}
\fill[green!30] (0,0) -- (1.5,2.5) -- (5,2) -- (4,-1) -- cycle;
\draw (0,0) -- (1.5,2.5) -- (5,2) -- (4,-1) -- cycle;
\node (f) at (2.5, 1) {$\sF$};
\end{tikzpicture}
\begin{block}{Integer linear programming}
\begin{align*}
  \sC = \{ M \in \pZ^{\ns \times \nt} : SM = T \}
\end{align*}
{\bf Proposition.}
Let $v$ be a random vector.
\vspace{-2cm}
\begin{columns}[t]
	\column{.5\textwidth}
	\centering
	\begin{align*}
	{\bf min.} &\ xMv \\
	{\bf s.t.} &\ SM=T\\
	&\ M \succeq 0
	\end{align*}
	\column{0.5\textwidth}
	\centering
	\begin{align*}
	{\bf max.} &\ xMv \\
	{\bf s.t.} &\ SM=T\\
	&\ M \succeq 0
	\end{align*}
\end{columns}
With probability 1, $x \in \sF$ iff both ILPs have same answer.

{\bf Computation. } Linear at training time, solving two ILPs at test time
\end{block}


\begin{block}{Linear programming}
\begin{align*}
\label{eqn:consistentLP}
\sClp &\eqdef \{ M \in \pR^{\ns \times \nt} \mid \SF M = \TF \} \\
\end{align*}
{\bf Proposition.}
	\label{prop:projection}
Let $M_1$ and $M_2$ be two ``random enough'' mappings inside $\sClp$.
With probability 1, $x \in \sFlp$ iff $x M_1 = x M_2$.

%{\bf Algorithm. }We first find $M_1$ in the relative interior of $\sClp$ by solving a linear program. 
%We then choose a random point on the ball around $M_1$ and project it back to $\sClp$.

{\bf Computation. } Solving one LP at training time, linear at test time
\end{block}

\begin{block}{Linear system}

\begin{align*}
\sCls &\eqdef \{ M \in \R^{\ns \times \nt} \mid \SF M = \TF \}
\end{align*}

{\bf Proposition.}  The vector $x$ is in row space of $S$ iff $x \in \sFls$.
	


A linear combination of training examples:\\
\centering
 {\scriptsize M(\nl{area of Ohio}) = M(\nl{area of Iowa}) + M(\nl{cities in Ohio}) - M(\nl{cities in Iowa})}
\newcommand{\scc}{0.65}
\begin{center}
\begin{tikzpicture}

		\node[scale=\scc] (S) at (0,0) {S =  $\begin{blockarray}{lccccccc}
		&\matindex{\nl{area}} & \matindex{\nl{of}} & \matindex{\nl{Ohio}} & \matindex{\nl{cities}} & \matindex{\nl{in}} & \matindex{\nl{Iowa}}&\\	
		\begin{block}{l[cccccc]c}
		\matindex{\nl{area of Iowa}} & 1 & 1 & 0 & 0 & 0 & 1 & \color{red} 1\\
		\matindex{\nl{cities in Ohio}} & 0 & 0 & 1 & 1 & 1 & 0 &\color{red} 1\\
		\matindex{\nl{cities in Iowa}} & 0 & 0 & 0 & 1 & 1 & 1 &\color{red} -1\\
		\end{block}
		\end{blockarray}$};
	
		\node[scale=\scc,below] (Se) at (S.south) { $\begin{blockarray}{lccccccc}
			\color{white}&\matindex{\nl{area}} & \matindex{\nl{of}} & \matindex{\nl{Ohio}} & \matindex{\nl{cities}} & \matindex{\nl{in}} & \matindex{\nl{Iowa}}&\\		
			\begin{block}{l[cccccc]c}
			\matindex{{\large x=} \ \nl{area of Ohio}} & 1 & 1 & 1 & 0 & 0 & 0 &\color{white} 0\\
			\end{block}
			\end{blockarray}$};


		\node[scale=\scc] (T) at (13,0) {T = $\begin{blockarray}{lccccc}
		&\matindex{\wl{area}} & \matindex{\wl{city}} & \matindex{\wl{OH}} & \matindex{\wl{IA}} &\\	
		\begin{block}{l[cccc]c}
		\matindex{\wl{area(IA)}} & 1 & 0 & 0 & 1 & \color{red} 1\\
		\matindex{\wl{city(OH)}} & 0 & 1 & 1 & 0 & \color{red} 1\\ 
		\matindex{\wl{city(IA)}} & 0 & 1 & 0 & 1 & \color{red} -1\\
		\end{block}
		\end{blockarray}$};
	
	
		\node[below, scale=\scc] (T) at (T.south) { $\begin{blockarray}{lccccc}
			&\matindex{\wl{area}} & \matindex{\wl{city}} & \matindex{\wl{OH}} & \matindex{\wl{IA}}&\\	
			\begin{block}{l[cccc]c}
			\matindex{{\large y=} \ \wl{area(OH)}} & 1 & 0 & 1 & 0 &\color{white} 0\\
			\end{block}
			\end{blockarray}$};
	\end{tikzpicture}
\end{center}	


\end{block}
%{\bf Computation. } Linear in the training time two ILPs in the test time.


%\begin{block}{Dual view}
%\newcommand{\scale}{0.5}



\begin{tikzpicture}[ mapping/.style={circle,fill,scale = \scale}]

\node[mapping] (A0) at (8,0) {};
\node[mapping] (A1) at (8,1.5) {};
\node[mapping] (A2) at (10,1) {};
\node[mapping] (A3) at (6,5) {};
\node[mapping] (A4) at (9,0) {};
\node[mapping] (A5) at (6,2) {};
\node[mapping] (A6) at (11,4) {};
\node[mapping] (A7) at (8.5,4.5) {};
\node[mapping] (A8) at (8,3.5) {};
\node[mapping] (A9) at (8,2.5) {};
\node[mapping] (A10) at (6,0) {};
\node[mapping] (A11) at (11,1) {};

\node (AA) [draw,fit=(A0) (A1) (A2)
(A3) (A4) (A5) (A6) (A7) (A8) (A9)] {};



\node[mapping] (B0) at (0,0) {};
\node[mapping] (B1) at (0,1.5) {};
\node[mapping] (B2) at (0,3) {};
\node[mapping] (B3) at (0,4.5) {};

\node (trainingset) [draw,fit=(B0) (B1) (B2) (B3)]  {};

\draw[gray!25,-] (B0) -- (A1);
\draw[gray!25,-] (B0) -- (A9);
\draw[gray!25,-] (B0) -- (A8);

\draw[gray!25,-] (B1) -- (A1);
\draw[gray!25,-] (B1) -- (A9);
\draw[gray!25,-] (B1) -- (A8);

\draw[gray!25,-] (B2) -- (A1);
\draw[gray!25,-] (B2) -- (A9);
\draw[gray!25,-] (B2) -- (A8);

\draw[gray!25,-] (B3) -- (A1);
\draw[gray!25,-] (B3) -- (A9);
\draw[gray!25,-] (B3) -- (A8);


\draw[gray!25,-] (B0) -- (A2);
\draw[gray!25,-] (B1) -- (A7);
\draw[gray!25,-] (B2) -- (A4);
\draw[gray!25,-] (B3) -- (A3);
\draw[gray!25,-] (B1) -- (A0);
\draw[gray!25,-] (B0) -- (A0);
\draw[gray!25,-] (B2) -- (A3);


\node[yshift = 1cm] (description) at (trainingset.north) {Training set};

\node[yshift=1cm] (B) at (AA.north) {\Large $\sM$};
\node[circle, draw, fit=(A9) (A8) (A1),inner sep=0] (c)  {};
\node[yshift=0.4cm,xshift=1.2cm] (B) at (c.north) {\Large $\sC$};

\newcommand{\xx}{18}
\newcommand{\yy}{2}
\newcommand{\rr}{3}
\newcommand{\rb}{10}
\newcommand{\len}{5}

\newcommand{\x2}{0}
\newcommand{\y2}{-4}


\draw[thick] (\xx+\rr/2,\yy) circle (\rr);
\draw[thick] (\xx-\rr/2,\yy) circle (\rr);
\draw[thick] (\xx, \yy+2*\rr/3) circle (\rr);
\begin{scope}[fill = green!30]
    \clip (\xx+\rr/2,\yy) circle (\rr);
    \clip (\xx-\rr/2,\yy) circle (\rr);
	\clip (\xx, \yy+2*\rr/3) circle (\rr);
\fill (\xx+\rr/2,\yy) circle (\rr);
\end{scope}


\node[yshift=0.5cm] (F) at (\xx,\yy) {\Large $\sF$};
\end{tikzpicture}
%\footnotesize Larger training set $\implies$ fewer consistent mappings $\implies$ Larger safe set
%\end{block} 



%----------------------------------------------------------------------------------------

\end{column} % End of column 2.2



\begin{column}{\sepwid}\end{column} % Empty spacer column

\begin{column}{\onecolwid} % The third column

\begin{block}{Details}
\begin{itemize}
\item {\bf Source atoms:} Replace words with $n$-grams to handle polysemy
\item {\bf Target atoms:} Add ordering information to predicates to reconstruct logical forms
\item {\bf Removing noise:} Use a relax constraint, $\|SM-T\|_1 \le \gamma$, instead of $SM=T$
\end{itemize}
\end{block}

\begin{block}{Other applications}
\begin{itemize}
\item {\bf Active learning:}
Choose linearly independent sentences to be annotated

\item {\bf Paraphrasing:}
Find all sentences that yield the same thing under all consistent semantic mappings

\item {\bf Learning from denotations:}
 Training data consists of (question, answer) pairs

\end{itemize}
\end{block}

\begin{block}{Results}
\begin{itemize}
{\bf \item[] Artificial dataset}\\
\begin{itemize}
\item[] Input/output vocabulary size is 70.
\item[] {\scriptsize \nl{w34, w22, w17, w12} $\rightarrow$
 \wl{p10,p15,p10,p20,p40,p47}}\\
\end{itemize}
\vspace{1cm}
\pgfplotsset{compat=1.6} 
\newcommand{\pOne}{0.2}
\newcommand{\pTwo}{0.5}
\newcommand{\pThree}{0.7}
\newcommand{\nTrial}{100}
%\pgfplotsset{compat=1.11}

\pgfplotsset{
x tick label style={/pgf/number format/fixed,
        /pgf/number format/precision=1,
		font=\tiny},
y tick label style={/pgf/number format/fixed,
        /pgf/number format/precision=1,
		font=\tiny},
y label style={font=\tiny},
x label style={font=\tiny},
legend style={font=\fontsize{10}{5}\selectfont},
   }%
\begin{tabular}{ll}
\begin{tikzpicture}
   \begin{axis}[
   xmax=1.1,xmin=0,
   ymin= 0,ymax=1.1,
   xlabel=\emph{Fraction of data},
ylabel=\emph{Recall},
   xtick={0,0.3,0.6,...,1},
   ytick={0,0.2,0.4,...,1},
legend style={legend pos=south east},
 ymajorgrids=true,
    grid style=dashed,
legend cell align=left,
   ]
\addplot coordinates{
(0.1,1.0)
(0.2,1.0)
(0.3,1.0)
(0.4,1.0)
(0.5,1.0)
(0.6,1.0)
(0.7,1.0)
(0.8,1.0)
(0.9,1.0)
(1,1.0)
};
\addplot coordinates{
(0.1,0.0)
(0.2,0.08)
(0.3,0.5)
(0.4,0.56)
(0.5,0.56)
(0.6,0.56)
(0.7,0.62)
(0.8,0.94)
(0.9,0.94)
(1,1.0)
};
\addplot coordinates{
(0.1,0.0)
(0.2,0.44)
(0.3,0.56)
(0.4,0.62)
(0.5,0.68)
(0.6,1.0)
(0.7,1.0)
(0.8,1.0)
(0.9,1.0)
(1,1.0)
};
\addplot coordinates{
(0.1,0.04)
(0.2,0.46)
(0.3,0.58)
(0.4,0.66)
(0.5,0.72)
(0.6,1.0)
(0.7,1.0)
(0.8,1.0)
(0.9,1.0)
(1,1.0)
};



    \legend{ precision (all), recall (ILP), recall (LP), recall (LS)}
    \end{axis}
    \end{tikzpicture}&
\begin{tikzpicture}
   \begin{axis}[
   xmax=152,xmin=0,
   ymin= 0,ymax=1.1,
   xlabel=number of mistakes,
ylabel=\emph{Recall},
%    y label style={at={(axis description cs:0.15,0.5)},anchor=south},
   xtick={0,30,60,...,150},
   ytick={0,0.2,0.4,...,1},
legend style={legend pos=south west,font=\fontsize{10}{5}\selectfont},
 ymajorgrids=true,
    grid style=dashed,
legend cell align=left,
   ]
\addplot coordinates{
	(0,    1.0)
	(30,    1.0)
	(60,    1.0)
	(90,    1.0)
	(120,    1.0)
	(150,    1.0)

};

\addplot coordinates{
	(0,    1.0)
	(30,    1.0)
	(60,    0.84)
	(90,    0.28)
	(120,    0.28)
	(150,    0.0)
};
    \legend{precision (ILP), recall (ILP)}
    \end{axis}
    \end{tikzpicture}\\



   \begin{tikzpicture}
   \begin{axis}[
   xmax=1.1,xmin=0,
   ymin= 0.3,ymax=1.1,
   xlabel=\emph{Recall},
ylabel=\emph{Precision},
 %   y label style={at={(axis description cs:0.15,0.5)},anchor=south},
   xtick={0,0.2,0.4,...,1},
   ytick={0,0.2,0.4,...,1},
legend style={legend pos=south east,font=\fontsize{10}{5}\selectfont},
 %   legend pos= south east,
    legend columns=2, 
 ymajorgrids=true,
legend cell align=right,
    grid style=dashed,   
   ]
\addplot coordinates{
(0.04,1.0)
};
\addplot coordinates{
(0.04,0.5)
(0.1,0.5555555555555556)
(0.24,0.5714285714285714)
(0.26,0.5)
(0.28,0.4375)
(0.3,0.3488372093023256)
};
\addplot coordinates{
(0.48,1.0)
};
\addplot coordinates{
(0.48,0.8571428571428571)
(0.48,0.8571428571428571)
(0.5,0.8333333333333334)
(0.5,0.8064516129032258)
(0.56,0.717948717948718)
(0.64,0.64)
};
\addplot coordinates{
(0.78,1.0)
};
\addplot coordinates{
(0.78,0.9512195121951219)
(0.78,0.9512195121951219)
(0.78,0.9512195121951219)
(0.78,0.9069767441860465)
(0.82,0.8913043478260869)
(0.84,0.84)
};
  \legend{
	$\pointM$(\pOne), 
 	$\ourM$\ (\pOne), 
 	$\pointM$(\pTwo), 
 	$\ourM$\ (\pTwo),
 	$\pointM$(\pThree), 
 	$\ourM$\ (\pThree)}
    \end{axis}
\end{tikzpicture}&
   \begin{tikzpicture}
   \begin{axis}[
   xmax=1.1,xmin=0,
   ymin= 0.3,ymax=1.1,
   xlabel=\emph{Recall},
ylabel=\emph{Precision},
 %   y label style={at={(axis description cs:0.15,0.5)},anchor=south},
   xtick={0,0.2,0.4,...,1},
   ytick={0,0.2,0.4,...,1},
legend style={legend pos=south east,font=\fontsize{10}{5}\selectfont},
    legend columns=2, 
 ymajorgrids=true,
legend cell align=left,
    grid style=dashed,   
   ]

\addplot coordinates{
(0.04,1.0)
};
\addplot coordinates{
(0.2,1.0)
(0.2,1.0)
(0.2,0.9090909090909091)
(0.22,0.7857142857142857)
(0.24,0.5454545454545454)
(0.24,0.35294117647058826)
};
\addplot coordinates{
(0.52,1.0)
};
\addplot coordinates{
(0.62,1.0)
(0.62,1.0)
(0.7,1.0)
(0.7,0.9459459459459459)
(0.7,0.8974358974358975)
(0.72,0.8372093023255814)
};
\addplot coordinates{
(0.78,1.0)
};
\addplot coordinates{
(0.8,1.0)
(0.8,1.0)
(0.8,1.0)
(0.82,1.0)
(0.82,0.9111111111111111)
(0.84,0.84)
};
  \legend{
	$\pointM$(\pOne), 
 	$\ourM$\ (\pOne), 
 	$\pointM$(\pTwo), 
 	$\ourM$\ (\pTwo),
 	$\pointM$(\pThree), 
 	$\ourM$\ (\pThree)}
  \end{axis}
\end{tikzpicture}
\end{tabular}


\item[] {\bf GeoQuery dataset}\\
\begin{itemize}
\item[] 880 (question, logical form) pairs
\item[] {\scriptsize \nl{how long is the mississippi} $\rightarrow$
 \wl{(answer(len(riverid mississippi)))}}\\

%\item[] 84\% had a unique way of reconstructing logical form
\end{itemize}
\vspace{1cm}
\pgfplotsset{
x tick label style={/pgf/number format/fixed,
        /pgf/number format/precision=1,
		font=\tiny},
y tick label style={/pgf/number format/fixed,
        /pgf/number format/precision=1,
		font=\tiny},
y label style={font=\tiny},
x label style={font=\tiny},
legend style={font=\fontsize{10}{5}\selectfont},
   }%
\begin{tabular}{ll}
\begin{tikzpicture}
 \begin{axis}[
   xmax=1.1,xmin=0,
   ymin= 0,ymax=1.1,
   xlabel=\emph{Fraction of data},
ylabel=\emph{Recall},
   xtick={0,0.3,0.6,...,1},
   ytick={0,0.2,0.4,...,1},
legend style={legend pos=south east},
 ymajorgrids=true,
    grid style=dashed,
legend cell align=left,
   ]
\addplot coordinates{
(0.1,1.0)
(0.2,1.0)
(0.30000000000000004,1.0)
(0.4,1.0)
(0.5,1.0)
(0.6,1.0)
(0.7,1.0)
(0.7999999999999999,1.0)
(0.8999999999999999,1.0)
(0.9999999999999999,1.0)
};

\addplot coordinates{
(0.1,0.15357142857142858)
(0.2,0.2571428571428571)
(0.30000000000000004,0.3392857142857143)
(0.4,0.41785714285714287)
(0.5,0.5142857142857142)
(0.6,0.5464285714285714)
(0.7,0.6)
(0.7999999999999999,0.6642857142857143)
(0.8999999999999999,0.6642857142857143)
(0.9999999999999999,0.7)
};
    \legend{precision (LS), recall (LS)}    \end{axis}
    \end{tikzpicture}&
\begin{tikzpicture}
    \begin{axis}[
   xmax=1.1,xmin=0,
   ymin= 0,ymax=1.1,
   xlabel=\emph{Fraction of data},
ylabel=\emph{Recall},
   xtick={0,0.3,0.6,...,1},
   ytick={0,0.2,0.4,...,1},
legend style={legend pos=south east},
 ymajorgrids=true,
    grid style=dashed,
legend cell align=left,
   ]
\addplot coordinates{
(0.1,0.18214285714285713)
(0.2,0.32142857142857145)
(0.30000000000000004,0.46785714285714286)
(0.4,0.6)
(0.5,0.6857142857142857)
(0.6,0.7)
(0.7,0.7)
(0.7999999999999999,0.7)
(0.8999999999999999,0.7)
(0.9999999999999999,0.7)
};

\addplot coordinates{
(0.1,0.15357142857142858)
(0.2,0.2571428571428571)
(0.30000000000000004,0.3392857142857143)
(0.4,0.41785714285714287)
(0.5,0.5142857142857142)
(0.6,0.5464285714285714)
(0.7,0.6)
(0.7999999999999999,0.6642857142857143)
(0.8999999999999999,0.6642857142857143)
(0.9999999999999999,0.7)
};



    %\legend{Independent examples, Random examples}
    \legend{Active learning, Passive learning}
  \end{axis}
    \end{tikzpicture}
\end{tabular}

\end{itemize}
\end{block}

%----------------------------------------------------------------------------------------

\end{column} % End of the third column

\end{columns} % End of all the columns in the poster

\end{frame} % End of the enclosing frame

\end{document}
