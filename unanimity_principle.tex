\begin{figure}
	\newcommand{\scale}{0.6}
	\tikzset{
    >=stealth',
    inputBox/.style={
           rectangle,
			inner sep=0,
           rounded corners=0.1em,
           draw=black,
           text centered,
			scale=\scale
},
   mappingBox/.style={
           rectangle,
			inner sep=0,
           rounded corners=0.3em,
           draw=black,
			fill=gray!30,
           text centered,
			scale=\scale
},
    pil/.style={
line width=6,
           ->,
           thick,
		}
}

		
			\begin{tikzpicture}%[scale=\scale]
			
			
			
			\def\x{4.4}
			\def\y{3}
			\def\yy{2.5}
			
			
			\def\spx{0.2}
			\def\spy{0.2}
			
			
			\node (trainings)[inputBox] at (1.5*\x,\y+\yy+4*\spy){
				\begin{tabular}{l|l}
				\bf input & \bf output \\ 	\hline 
				\nl{area of Iowa}& \wl{area(IA)}\\
				\nl{cities in Ohio}&  \wl{city(OH)} \\
				\nl{cities in Iowa} & \wl{city(IA)}\\

				\end{tabular}
			};
			
			\node (A)[mappingBox] at (0,\yy){
				\begin{tabular}{l}
				\bf mapping 1 \\ 	
				\end{tabular}
			};


			
			\node (B) [mappingBox]at (\x,\yy){
				\begin{tabular}{l}
				\bf mapping 2 \\				
				\end{tabular}
			};
			
			\node (C) [mappingBox]at (2*\x+4*\spx,\yy){
				\begin{tabular}{l}
				 \bf mapping k \\ 
				\end{tabular}
			};

				\node (AA)[inputBox] at (0,0){
				\begin{tabular}{l}
				\bf output 1 \\ \hline
				\small\wl{area(OH)}\\
				\small\wl{area(OH)}\\
				%\small\wl{city} \\
				\end{tabular}
			};
			
			\node (BB) [inputBox]at (\x,0){
				\begin{tabular}{l}
				\bf output 2 \\ \hline
				\small\wl{area(OH)}\\
				\small\wl{area(OH)}\\
				%\small\wl{city} \\
				\end{tabular}
			};
			
			\node (CC) [inputBox]at (2*\x+4*\spx,0){
				\begin{tabular}{l}
				\bf output k \\ \hline
				\small\wl{area(OH)}\\
				\small\wl{OH}\\
				%\small\wl{city} \\
				\end{tabular}
			};
			
			
			\node [left] at (trainings.west) {\tiny training examples};
			
			
			\node (testingsInput)[inputBox] at (-1*\x-3*\spx-\spx,0){
				\begin{tabular}{l}
				\bf input \\ \hline
				\small\nl{area of Ohio}\\ 
				\small\nl{Ohio area}\\ 
				%\small\nl{cities in}\\ \hline
				\end{tabular}
			};
			
			\node (testingsOutput)[inputBox] at (4*\x+4*\spx,0){
				\begin{tabular}{l}
				\bf output \\ \hline
				\small\wl{area(Ohio)}\\ 
				\small don't know \\ 
				%\small \wl{city} \\ \hline
				\end{tabular}
			};
			
			\node [above] at (testingsInput.north) {\tiny testing examples};
			
			%\node [above] at (testingsOutput.north) {\tiny test examples};
			
			\def\checkmark{\tikz\fill[fill=green,scale=1.5](0,.35) -- (.25,0) -- (1,.7) -- (.25,.15) -- cycle;} 
			
			\newcommand{\Cross}{$\mathbin{\tikz [line width=.5ex, red,scale=0.75] \draw (0,0) -- (1,1) (0,1) -- (1,0);}$}%
			
			\node (testingsResults)[scale=\scale] at (3*\x + 2*\spx,0){
				\begin{tabular}{c}
				\small \bf unanimity \\ 
				\checkmark\\
				\Cross\\
			%	\checkmark\\
				\end{tabular}
			};
			
			\node (3dots)[] at (1.5*\x+2*\spx,\yy){...};
			\node (3dots)[] at (1.5*\x+2*\spx,0){...};
			
    \node[fit={(-\x/2-\spx,-\y/2-2*\spy) (3.6*\x,\y/2+\yy-\spy)}, inner sep=0pt, draw=black,rounded corners=.3em] (rect) {};

		%	\node (rec)[scale=\scale] { (-\x/2,-\y/2) rectangle (3.6*\x,\y/2+\yy)};
			
			\draw[pil, shorten >=3pt] (rect.north) -- (A.north);
			\draw[pil, shorten >=2pt] (rect.north) -- (B.north);
			\draw[pil, shorten >=3pt] (rect.north) -- (C.north);
			
			\draw[pil] (trainings.south) -- (rect.north);
			
			\draw[pil] (3.6*\x,0) -- (testingsOutput.west);
			\draw[pil] (testingsInput.east) -- (-\x/2-\spx, 0);
			
			
			\draw[double,-implies] (A.south) -- (AA.north);
			\draw[double,-implies] (B.south) -- (BB.north);
			\draw[double,-implies] (C.south) -- (CC.north);
			
		

\end{tikzpicture}

\end{figure}
